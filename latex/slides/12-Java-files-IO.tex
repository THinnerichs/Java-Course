\input{../templates/course_definitions}
% This Document contains the information about this course.

% Authors of the slides
\author{Vincent Gerber, Tilman Hinnerichs}

% Name of the Course
\institute{Java Kurs}

% Fancy Logo
\titlegraphic{\hfill\includegraphics[height=1.25cm]{../templates/fsr_logo_cropped}}


\usepackage{qtree}

\title{Java}
\subtitle{Java files IO}
\date{\today}

\usepackage{color}

\begin{document}
\begin{frame}
	\titlepage
\end{frame}

\begin{frame}{Overview}
	\tableofcontents
\end{frame}

\section{How to use streams}
\begin{frame}{Streams}
	A stream can be defined as a sequence of data. We got two kinds of streams in Java:
	\vfill
	\begin{tabular}{rl}
		InPutStream&is used to read data from a source\\
		OutPutStream&is used for writing data to a destination
	\end{tabular}
	\vfill
	Many implementations can be found in the module \textbf{java.io}.
\end{frame}

\begin{frame}[fragile]{File Streams}
	Byte streams used to perform input and ouput of 8-bit bytes. The most common streams for file reading and writing are \textbf{FileInputStream} and \textbf{FileOutputStream}:\\
	
\end{frame}

\begin{frame}[fragile]{How to use file streams}
	We would like to copy a file "trateotu.txt":
	\begin{lstlisting}
	import java.io.*
	... main(String args[]) throws IOException {
		FileInputStream in = null;
		FileOutputStream out = null;
		
		try {
			in = new FileInputStream("trateotu-input.txt");
			out = new FileOutputStream("trateotu-output.txt");
			
			int c;
			while ((c = in.read()) != -1) {
				out.write(c);
			}
		}
		finally {
			if (in != null) {
				in.close();
			}
			if (out != null) {
				out.close();
			}
		}
	}
	\end{lstlisting}
\end{frame}


\begin{frame}{Other useful methods for InputStream}
	\begingroup
	\def\arraystretch{1.5}
	\begin{tabular}{rp{8cm}}
		public void close()&releases any system resources associated with the file\\
		protected void finalize()&cleans up connection to the file, closes stream when there are no more references\\
		public int read(int r)&this method reads the specified byte of data\\
		public int available()& returns the amount of bytes which can be read from the input stream
	\end{tabular}
	\endgroup
\end{frame}

\begin{frame}{Othe useful methods for OutputStream}
	
	\begingroup
	\def\arraystretch{1.5}
	\begin{tabular}{rp{8cm}}
		public void close()&closes output stream, releases any resources\\
		public void finalize()&see InputStream\\
		public void write(int w)&writes the specified byte
	\end{tabular}
	\endgroup
\end{frame}

\begin{frame}[fragile]{Character Streams}
	Instead of the previously used Byte streams, we can also use \textbf{Character streams}. \textbf{Character streams} are able to perform input and output for 16-bit unicode:
	
	\begin{lstlisting}
		in = new FileReader("ltuae-input.txt);
		out = new FileReader("ltuae-output.txt);
	\end{lstlisting}
\end{frame}

\begin{frame}{User interaction}
	Now knowing about streams we can also perform command line input/output. Java provides the following three streams, given by your OS:\\
	\vfill
	\begingroup
	\def\arraystretch{1.5}
	\begin{tabular}{r p{8cm}}
		Standard Input&is used to feed data into you program, is your standard input stream, represented as \textbf{System.in}\\
		Standard Output & is used to output the data produced by the user's program, represented as \textbf{System.out}\\
		Standard Error&is used to output the error data produced by our program, \textbf{System.err}
	\end{tabular}
	\endgroup
\end{frame}

\begin{frame}[fragile]{First Console IO program}
	We will now read standard input stream until the user types a "q":\\
	\begin{lstlisting}
		InputStreamReader cin = null;
		
		try {
			cin = new InputStreamReader(System.in);
			System.out.println("Enter characters, 'q' to quit.");
			char c;
			do {
				c = (char) cin.read();
				System.out.print(c);
			} while(c != 'q');
		}
		finally {
			if (cin != null) {
				cin.close();
			}
		}
	\end{lstlisting}
\end{frame}

\end{document}