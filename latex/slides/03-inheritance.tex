\input{../templates/course_definitions}
% This Document contains the information about this course.

% Authors of the slides
\author{Vincent Gerber, Tilman Hinnerichs}

% Name of the Course
\institute{Java Kurs}

% Fancy Logo
\titlegraphic{\hfill\includegraphics[height=1.25cm]{../templates/fsr_logo_cropped}}


\title{Java}
\subtitle{Introduction to OOP and inheritance}
\date{\today}

\begin{document}

\section{OOP in Java}

\begin{frame}{}
	\begin{center}
		{\huge Object Oriented Programming}
	\end{center}
\end{frame}

\subsection{General information}

\begin{frame}[fragile]{What is OOP?}
	We have a lot of objects around us\\
	\begin{itemize}
		\item They have a state and a behaviour
		\item All have a building plan
	\end{itemize}
	
	Classes $ \rightarrow $ building plan\\
	Objects $ \rightarrow $ entity in real world build from building plan
	
\end{frame}

\begin{frame}[fragile]{How to build a class}
	\begin{lstlisting}
		public class MyClass {
			//Constructor to create the Object
			public MyClass(){
			...
			}
			
			//Attributes to describe state
			private int myAttribute;
			...
			
			//Methods to describe behaviour
			public MyMethod(int myParameter, ...) {
				this.myAttribute = myParameter;
				...
			}
		}
	\end{lstlisting}
\end{frame}

\begin{frame}[fragile]{How to build a student?}
	Attributes describing a student? \\
	\pause 
	\begin{itemize}
		\item Name
		\item Matriculation number
		\item Age
	\end{itemize}	
	\pause
	Methods describing its behaviour?
	\pause
	\begin{itemize}
		\item Change name
		\item Increase age
		\item Calculate 1+1
	\end{itemize}
\end{frame}

\begin{frame}[fragile]{Class Student}
	\begin{lstlisting}
	public class Student {
	//Constructor
	public Student (String name, int matriculationNumber) {
	this.name = name;
	this.matriculationNumber = matriculationNumber;
	}
	
	// Attributes
	private String name; 
	private int matriculationNumber; 
	
	
	// Methods
	public void setName(String name) {
	this.name = name;
	}
	
	public int getMatriculationNumber() {
	return matriculationNumber;
	}
	
	}
	\end{lstlisting}
	
	% What is visible here:
	% Attributes store the state of the object
	% Methods implement the behaviour of the object
	
\end{frame}

\begin{frame}[fragile]{Creation}
	We learned how to declare and assign a primitive datatype.}

\begin{lstlisting}
int a; // declare a
a = 273; // assign 273 to a
\end{lstlisting} 

The creation of an object works similar.

\begin{lstlisting}
Student example = new Student(); 
// create an instance of Student
\end{lstlisting}
The \textbf{object} derived from a \textbf{class} is also called \textbf{instance}.
The variable is called the \textbf{reference}.
\end{frame}

\subsection{Methods}
\begin{frame}[fragile]{Calling a Method}
\begin{lstlisting}
public class Student {

private String name;

public String getName() {
return name;
}
public void setName(String newName) {
name = newName;
}
public void printName(){
System.out.println(this.name);
}
}
\end{lstlisting}
The class \emph{Student} has three methods: \emph{void printName()}.
\end{frame}

\begin{frame}[fragile]{Calling a Method}
\begin{lstlisting}
public class Main {

public static void main(String[] args) {
Student example = new Student(); // creation
example.setName("Jane"); // method call
String name = example.getName(); 
System.out.println(name); // Prints "Jane"

example.printName();
}

}
\end{lstlisting}
You can call a method of an object after its creation with \textbf{reference.methodName();}.
\end{frame}

\begin{frame}[fragile]{Calling a Method}
\begin{lstlisting}
public class Student {

private String name;

public void setName(String newName) {
name = newName;
printName();   // Call own method
this.printName(); // Or this way
}

public void printName() {
System.out.println(name);
}

}
\end{lstlisting}
You can call a method of the own object by simply writing \textbf{methodName();} or \textbf{this.methodName();}
\end{frame}

\begin{frame}[fragile]{Methods with Arguments}

\begin{lstlisting}
public class Calc {

public void add(int summand1, int summand2) {
System.out.println(summand1 + summand2);
}

public static void main(String[] args) {
int summandA = 1;
int summandB = 2;
Calc calculator = new Calc();
System.out.print("1 + 2 = ");
calculator.add(summandA, summandB); 
// prints: 3
}

}
\end{lstlisting}
\end{frame}

\subsection{Return Value}
\begin{frame}[fragile]{Methods with Return Value}
A method without a return value is indicated by \textbf{void}:
\begin{lstlisting}
public void add(int summand1, int summand2) {
System.out.println(summand1 + summand2);
}
\end{lstlisting}
A method with an \textbf{int} as return value:
\begin{lstlisting}
public int add(int summand1, int summand2) {
return summand1 + summand2;
}
\end{lstlisting}
% TODO explain return statement
\end{frame}

\begin{frame}[fragile]{Calling Methods with a return value}
\begin{lstlisting}
public class Calc {

public int add(int summand1, int summand2) {
return summand1 + summand2;
}

public static void main(String[] args) {
Calc calculator = new Calc();
int sum = calculator.add(3, 8);
System.out.print("3 + 8 = " + sum); 
// prints: 3 + 8 = 11
}

}
\end{lstlisting}
\end{frame}

\subsection{Constructor}

\begin{frame}[fragile]{Constructors}
\begin{lstlisting}
public class Calc {

private int summand1;
private int summand2;

public Calc() {
summand1 = 0;
summand2 = 0;
}

}
\end{lstlisting}
A constructor gets called upon creation of the object
\end{frame}

\begin{frame}[fragile]{Constructors with Arguments}
\begin{lstlisting}
public class Calc {

private int summand1;
private int summand2;

public Calc(int x, int y) {
summand1 = x;
summand2 = y;
}

}
\end{lstlisting}
\begin{lstlisting}
[...]
Calc myCalc = new Calc(7, 9);
\end{lstlisting}

A constructor can have arguments as well!
\end{frame}

\section{Conclusion}
\subsection{An Example}

\begin{frame}{Let's build a car}
Create a car with doors, wheels, gas, seats... \\
\vspace{1em}
Focus on:
\begin{description}
	\item[ID] unique id for each car
	\item[Car] gas, speed
	\item[Doors] can open/close
	\item[Wheels] air pressure, size,...
	\item[Seats] free, quality
	\item[\dots]
\end{description}
% 	The more you think about it, the more complex this program becomes.
% 	Focus on the relevant things.
%	Think about how the objects can be in relation, this will be discussed later
%	Show prepared classes in Java
\end{frame}


\begin{frame}[fragile]{Class Student}
\begin{lstlisting}
public static void main(String[] args) {
Student peter = new Student();
peter.changeName("Peter");
}
\end{lstlisting}
\end{frame}

\begin{frame}
	\titlepage
\end{frame}
\begin{frame}{Overview}
	\setbeamertemplate{section in toc}[sections numbered]
	\tableofcontents
\end{frame}

\section{Visibilities}
\begin{frame}[fragile]{Visibilities}
	\begin{itemize}
		\item public
		\item private
		\item protected
	\end{itemize}
\end{frame}
	
\begin{frame}[fragile]{Visibilities}

	\begin{lstlisting}
	
		public class Student {
			public String getName() {
				return "Peter";
			}
			
			private String getFavouriteFilm() {
				return "...";
			}
		}
	
		// [...]
		exampleStudent.getName(); // Works!
		exampleStudent.getFavouriteFilm(); // Error
	
	\end{lstlisting}
	
\end{frame}


\section{Arrays}
\begin{frame}[fragile]{Array}
	An array is a data-type that can hold a \textbf{fixed number} of elements. 
	An Element can be any simple data-type or object.
	\begin{lstlisting}
	public static void main(String[] args) {
	
	    int[] intArray = new int[10];
	    intArray[8] = 7; // assign 7 to the 9th element
	    intArray[9] = 8; // assign 8 to the last element
	    
	    System.out.println(intArray[8]); // prints: 7
	}
	\end{lstlisting}
	You can access every element via an index. A n-element array has indexes from 0 to (n-1).
\end{frame}

\begin{frame}[fragile]{Array Initialization} % AE
	You can initialize an array with a set of elements.
	\begin{lstlisting}
	public static void main(String[] args) {
	
	    int[] intArray = {3, 2, 7};
	    
	    System.out.println(intArray[0]); // prints: 3
	    System.out.println(intArray[1]); // prints: 2
	    System.out.println(intArray[2]); // prints: 7
	}
	\end{lstlisting}
\end{frame}

\begin{frame}[fragile]{Alternative Declaration}
	There two possible positions for the square brackets. 
	%I would recommend the first version to improve readability.
	\begin{lstlisting}
	public static void main(String[] args) {

	    // version 1	
	    int[] intArray1 = new int[10];
	    
	    // version 2
	    int intArray2[] = new int[10];
	}
	\end{lstlisting}
\end{frame}

\subsection{Multi-Dimensional Array}
\begin{frame}[fragile]{2-Dimensional Array}
	Arrays work with more than one dimension. 
	An m-dimensional array has m indexes for one element.
	\begin{lstlisting}
	public static void main(String[] args) {

	    // an array with 100 elements
	    int[][] intArray = new int[10][10];
	    
	    intArray[0][0] = 0;
	    intArray[0][9] = 9;
	    intArray[9][9] = 99;
	}
	\end{lstlisting}
\end{frame}

\begin{frame}[fragile]{Assignment with Loops}
	Loops are often used to assign elements in arrays.
	\begin{lstlisting}
	public static void main(String[] args) {

	    int[][] intArray = new int[10][10];
	    
	    for(int i = 0; i < 10; i++) {
	        for(int j = 0; j < 10; j++) {
	            intArray[i][j] = i*10 + j;
	        }
	    }
	}
	\end{lstlisting}
\end{frame}

\begin{frame}[fragile]{Arrays with objects}
	Loops are often used to assign elements in arrays.
	\begin{lstlisting}
	public static void main(String[] args) {

	    Student[][] studentArray = new Student[10][10];
	    
	    for(int i = 0; i < 10; i++) {
	        for(int j = 0; j < 10; j++) {
	            intArray[i][j] = new Student();
	        }
	    }
	}
	\end{lstlisting}
\end{frame}

\section{Inheritance}
\subsection{Inheritance}


\begin{frame}[fragile]{A special Delivery}
	Our class \emph{Letter} is a kind of \emph{Delivery} denoted by the keyword \textbf{extends}.
	\begin{itemize}
		\item \emph{Letter} is a \textbf{subclass} of the class \emph{Delivery}
		\item \emph{Delivery} is the \textbf{superclass} of the class \emph{Letter}
	\end{itemize}
	\begin{lstlisting}
	public class Letter extends Delivery {
	
	}
	\end{lstlisting}
	\vfill
	As mentioned implicitly above a class can has multiple subclasses. 
	But a class can only inherit directly from one superclass.
\end{frame}

\begin{frame}[fragile]{Example}
	We have the classes: \emph{PostOffice}, \emph{Delivery} and \emph{Letter}.
	They will be used for every example in this section and they will grow over time.
	\begin{lstlisting}
	public class Delivery {
	
	    private String address;
	    private String sender;
	    
	    public void setAddress(String addr) {
			address = addr;
	    }
	    
	    public void setSender(String snd) {
			sender = snd;
	    }
	    
	    public void printAddress() {
	        System.out.println(this.address);
	    }
	}
	\end{lstlisting}
\end{frame}

\begin{frame}[fragile]{Inherited Methods}
	The class \emph{Letter} also inherits all methods from the superclass \emph{Delivery}.
	\begin{lstlisting}
	public class PostOffice {
	
	    public static void main(String[] args) {
	    
	        Letter letter = new Letter();
	        
	        letter.setAddress("cafe ascii, Dresden");
	        
	        letter.printAddress();
	        // prints: cafe ascii, Dresden
	    }	
	}
	\end{lstlisting}
\end{frame}

\begin{frame}[fragile]{Override Methods}
	The method printAddress() is now additional definded in \emph{Letter}.
	\begin{lstlisting}[escapechar=!]
	public class Letter extends Delivery {
	
	    @Override
	    public void printAddress() {
	        System.out.println("a letter for " + this.address);    
	    }	
	}
	\end{lstlisting}
	% programer is AE
	\texttt{@Override} is an annotation. 
	It helps the programer to identify overwritten methods.
	It is not neccessary for running the code but improves readability.
	What annotations else can do we discuss in a future lesson.
\end{frame}
\begin{frame}[fragile]{Override Methods}
	Now the method \texttt{printAddress()} defined in \emph{Letter} will be used instead of the method defined
	in the superclass \emph{Delivery}.
	\begin{lstlisting}
	public class PostOffice {
	
	    public static void main(String[] args) {
	    
	        Letter letter = new Letter();
	        
	        letter.setAddress("cafe ascii, Dresden");
	        
	        letter.printAddress();
	        // prints: a letter for cafe ascii, Dresden
	    }	
	}
	\end{lstlisting}
\end{frame}

\subsection{Constructor}
\begin{frame}[fragile]{Super()}
	If we define a \textbf{constructor with arguments} in \emph{Delivery} we have to define a constructor
	with the same list of arguments in every subclass.
	\begin{lstlisting}[basicstyle=\ttfamily\scriptsize]
	public class Delivery {
	
	    private String address;
	    private String sender;
	    
	    public Delivery(String address, String sender) {
	        this.address = address;
	        this.sender = sender;
	    }
	    	    
	    public void printAddress() {
	        System.out.println(address);
	    }
	}
	\end{lstlisting}
\end{frame}

\begin{frame}[fragile]{Super()}
	For the constructor in the subclass \emph{Letter} we can use \texttt{super()} to call the constructor
	from the superclass.
	\begin{lstlisting}[escapechar=!]
	public class Letter extends Delivery {

	    public Letter(String address, String sender) {
	        super(address, sender);
	    }
	
	    @Override
	    public void printAddress() {
	        System.out.println("a letter for " + this.address);    
	    }	
	}
	\end{lstlisting}
\end{frame}

\begin{frame}[fragile]{Super() - Test}
	\begin{lstlisting}
	public class PostOffice {
	    
	    public static void main(String[] args) {	    
	        Letter letter = 
	            new Letter("cafe ascii, Dresden", "");
	        
	        letter.printAddress();
	        // prints: a letter for cafe ascii, Dresden
	    }
	}
	\end{lstlisting}
\end{frame}

\subsection{Implicit Inheritance}
\begin{frame}{Object}
	Every class is a subclass from the class \emph{Object}. 
	Therefore every class inherits methods from \emph{Object}.
	\vfill
	See \scriptsize\url{http://docs.oracle.com/javase/7/docs/api/java/lang/Object.html} \normalsize for
	a full reference of the class \emph{Object}.
\end{frame}

\begin{frame}[fragile]{toString()}
	\emph{Letter} is a subclass of \emph{Object}.
	Therefore \emph{Letter} inherits the method \texttt{toString()} from \emph{Object}.\\
	\texttt{System.out.println(argument)} will call \texttt{argument.toString()} to receive
	a printable String.
	\begin{lstlisting}[escapechar=!]
	public class PostOffice {
	    
	    public static void main(String[] args) {	    
	        Letter letter = 
	            new Letter("cafe ascii, Dresden", "");
	        
	        System.out.println(letter);
	        // prints: Letter@_some_HEX-value_
	        // for example: Letter@4536ad4d
	    }
	}
	\end{lstlisting}
\end{frame}

\begin{frame}[fragile]{Override toString()}
	\begin{lstlisting}[escapechar=!]
	public class Letter extends Delivery {

	    public Letter(String address, String sender) {
	        super(address, sender);
	    }
	
	    @Override
	    public String toString() {
	        return "a letter for " + this.address;
	    }	
	}
	\end{lstlisting}
\end{frame}

\begin{frame}[fragile]{Override toString() - Test}
	\begin{lstlisting}[escapechar=!]
	public class PostOffice {
	    
	    public static void main(String[] args) {	    
	        Letter letter = 
	            new Letter("cafe ascii, Dresden", "");
	        
	        System.out.println(letter);
	        // a letter for cafe ascii, Dresden
	    }
	}
	\end{lstlisting}
\end{frame}


\end{document}