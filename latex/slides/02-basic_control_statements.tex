\input{../templates/course_definitions}
% This Document contains the information about this course.

% Authors of the slides
\author{Vincent Gerber, Tilman Hinnerichs}

% Name of the Course
\institute{Java Kurs}

% Fancy Logo
\titlegraphic{\hfill\includegraphics[height=1.25cm]{../templates/fsr_logo_cropped}}


\title{Java}
\subtitle{Controll Statements and Functions}
\date{\today}

\begin{document}

\begin{frame}
	\titlepage
\end{frame}
\section{OOP in Java}

\begin{frame}{}
	\begin{center}
		{\huge Object Oriented Programming}
	\end{center}
\end{frame}

\subsection{General information}

\begin{frame}[fragile]{Class Student}
\begin{lstlisting}
public class Student {

	// Attributes
	private String name; 
	private int matriculationNumber; 
	
	
	// Methods
	public void setName(String name) {
		this.name = name;
	}
    
	public int getMatriculationNumber() {
		return matriculationNumber;
	}
    
}
\end{lstlisting}

% What is visible here:
% Attributes store the state of the object
% Methods implement the behaviour of the object

\end{frame}

\begin{frame}[fragile]{Creation}
	We learned how to declare and assign a primitive datatype.}
	
	\begin{lstlisting}
	    int a; // declare a
	    a = 273; // assign 273 to a
	\end{lstlisting} 
	
	The creation of an object works similar.
	
	\begin{lstlisting}
	    Student example = new Student(); 
		// create an instance of Student
	\end{lstlisting}
	The \textbf{object} derived from a \textbf{class} is also called \textbf{instance}.
	The variable is called the \textbf{reference}.
\end{frame}

\subsection{Methods}
\begin{frame}[fragile]{Calling a Method}
	\begin{lstlisting}
	public class Student {
    
		private String name;
	
		public String getName() {
        	return name;
    	}
		
	    public void setName(String newName) {
			name = newName;
	    }
	   
	}
	\end{lstlisting}
	The class \emph{Student} has two methods: \emph{void printTimetable()} and \emph{void printName()}.
\end{frame}

\begin{frame}[fragile]{Calling a Method}
\begin{lstlisting}
public class Main {

    public static void main(String[] args) {
        Student example = new Student(); // creation
        example.setName("Jane"); // method call
        String name = example.getName(); 
		System.out.println(name); // Prints "Jane"
    }
    
}
	\end{lstlisting}
	You can call a method of an object after its creation with \textbf{reference.methodName();}.
\end{frame}

\begin{frame}[fragile]{Calling a Method}
\begin{lstlisting}
	public class Student {
    
		private String name;
	
	    public void setName(String newName) {
			name = newName;
			printName();   // Call own method
			this.printName(); // Or this way
	    }
	    
	    public void printName() {
	        System.out.println(name);
	    }
        
	}
	\end{lstlisting}
	You can call a method of the own object by simply writing \textbf{methodName();} or \textbf{this.methodName();}
\end{frame}

\begin{frame}[fragile]{Methods with Arguments}

\begin{lstlisting}
public class Calc {

    public void add(int summand1, int summand2) {
        System.out.println(summand1 + summand2);
    }
	    
    public static void main(String[] args) {
        int summandA = 1;
        int summandB = 2;
        Calc calculator = new Calc();
        System.out.print("1 + 2 = ");
        calculator.add(summandA, summandB); 
		// prints: 3
    }
    
}
	\end{lstlisting}
\end{frame}

\subsection{Return Value}
\begin{frame}[fragile]{Methods with Return Value}
	A method without a return value is indicated by \textbf{void}:
	\begin{lstlisting}
	public void add(int summand1, int summand2) {
	    System.out.println(summand1 + summand2);
	}
	\end{lstlisting}
	A method with an \textbf{int} as return value:
	\begin{lstlisting}
	public int add(int summand1, int summand2) {
	    return summand1 + summand2;
	}
	\end{lstlisting}
	% TODO explain return statement
\end{frame}

\begin{frame}[fragile]{Calling Methods with a return value}
	\begin{lstlisting}
	public class Calc {
	
	    public int add(int summand1, int summand2) {
	        return summand1 + summand2;
	    }
	    
	    public static void main(String[] args) {
	        Calc calculator = new Calc();
	        int sum = calculator.add(3, 8);
	        System.out.print("3 + 8 = " + sum); 
			// prints: 3 + 8 = 11
	    }
        
	}
	\end{lstlisting}
\end{frame}

\subsection{Constructor}

\begin{frame}[fragile]{Constructors}
	\begin{lstlisting}
	public class Calc {
    
		private int summand1;
		private int summand2;
	
	    public Calc() {
			summand1 = 0;
			summand2 = 0;
	    }
        
	}
	\end{lstlisting}
	A constructor gets called upon creation of the object
\end{frame}

\begin{frame}[fragile]{Constructors with Arguments}
	\begin{lstlisting}
	public class Calc {
    
		private int summand1;
		private int summand2;
	
	    public Calc(int x, int y) {
			summand1 = x;
			summand2 = y;
	    }
        
	}
	\end{lstlisting}
	\begin{lstlisting}
	[...]
	Calc myCalc = new Calc(7, 9);
	\end{lstlisting}
    
	A constructor can have arguments as well!
\end{frame}

\section{Conclusion}
\subsection{An Example}

\begin{frame}{Let's build a car}
	Create a car with doors, wheels, gas, seats... \\
	\vspace{1em}
	Focus on:
	\begin{description}
		\item[ID] unique id for each car
		\item[Car] gas, speed
		\item[Doors] can open/close
		\item[Wheels] air pressure, size,...
		\item[Seats] free, quality
		\item[\dots]
	\end{description}
% 	The more you think about it, the more complex this program becomes.
% 	Focus on the relevant things.
%	Think about how the objects can be in relation, this will be discussed later
%	Show prepared classes in Java
\end{frame}

\end{document}