\input{../templates/course_definitions}
% This Document contains the information about this course.

% Authors of the slides
\author{Vincent Gerber, Tilman Hinnerichs}

% Name of the Course
\institute{Java Kurs}

% Fancy Logo
\titlegraphic{\hfill\includegraphics[height=1.25cm]{../templates/fsr_logo_cropped}}


\title{Java}
\subtitle{Collections part 2}
\date{\today}

\usepackage{color}

\begin{document}

\begin{frame}
\titlepage
\end{frame}

\begin{frame}{Overview}
\tableofcontents
\end{frame}

\section{Repetition}
\begin{frame}{Repetition}
	What we learned last time:
	\begin{itemize}
		\item How to use generics
		\item How to handle Javas lists, sets and iterators
	\end{itemize}
	What we will try to achieve today:
	\begin{itemize}
		\item How to use iterators on sets and lists
		\item How to use maps and what to with them
		\item What exceptions are and how to handle them
	\end{itemize}
\end{frame}

\begin{frame}{A quiz!}
	\begin{tabular}{c|c|c}
		&Set&List\\
		\hline
		\pause
		Same item twice in it? &&\\
		\hline
		Ordered?&&\\
		\hline
		Iterable?&&\\
		\hline
		What package to import &&\\
		\hline
		Declaring set type (variable type)&&\\
		\hline
		Building an instance (example)&&\\
		\hline
		Add an item&&\\
		\hline
		Removing an item&&\\
		
	\end{tabular}
\end{frame}

\begin{frame}{A quiz!}
	\begin{tabular}{c|c|c}
		&Set&List\\
		\hline
		Same item twice in it? &No!&Yes!\\
		\hline
		Ordered?&No!&Yes!\\
		\hline
		Iterable?&Yes!&Also yes!\\
		\hline
		What package to import &import java.util.*&import java.util.*\\
		\hline
		Declaring set type (variable type)&Set$<$T$>$ set&List$<$T$> $list\\
		\hline
		Building an instance (example)&= new HashSet$<$T$>$()& = new ArrayList$<$T$>$()\\
		\hline
		Add an item&set.add(item)&list.add(item)\\
		\hline
		Removing an item&set.remove(item)&list.remove(item)\\
	\end{tabular}
\end{frame}

\begin{frame}{Another quiz!}
	The iterator:\\
	\vspace{1cm}
	\begin{tabular}{l|l}
		&Iterator\\
		\hline
		How to declare&\\
		\hline
		How to build an instance&\\
		\hline
		First main function (With data type)&\\
		\hline
		Second main function (With data types)&\\
		\hline
		Third main function (With data type)&\\
		\hline
		How to get from collection?&\\
		
	\end{tabular}
\end{frame}

\begin{frame}[fragile]{How to iterate over sets and lists}
	\pause
	\begin{lstlisting}
		Set<T> mySet = new HashSet<T>();
		foreach(T item:mySet){
			item.doSomething();
		}
		
		List<T> myList = new ArrayList<T>();
		foreach(T item:myList){
		item.doSomething();
		}
	\end{lstlisting}
\end{frame}

\begin{frame}{Another quiz!}
	The iterator:\\
	\vspace{1cm}
	\begin{tabular}{l|l}
		&Iterator\\
		\hline
		How to declare&Iterator$<$T$>$ iter\\
		\hline
		How to build an instance&= new Iterator$<$T$>$()\\
		\hline
		First main function (With data type)&boolean iter.hasNext()\\
		\hline
		Second main function (With data types)&T iter.next()\\
		\hline
		Third main function (With data type)&T iter.remove()\\
		\hline
		How to get from collection(e.g. set)?&set.iterator()\\
		
	\end{tabular}
\end{frame}

\begin{frame}[fragile]{How to iterate over sets and lists using iterators}
	\begin{lstlisting}[basicstyle=\ttfamily\scriptsize]
	Set<T> mySet = new HashSet<T>();
	Iterator<T> myIter = mySet.iterator();
	
	while(myIter.hasNext()){
		T item = myIter.next();
		item.doSomething();
	}
	\end{lstlisting}
\end{frame}

\begin{frame}{Exercise}
	\begin{itemize}
		\item Create an array with 10 elements. Create a list and fill the list with the array 		elments. Create a set and fill the set with the list elments
		\item Extend our vending machine with an internal storage
	\end{itemize}
\end{frame}

\section{Maps and iterators}
\begin{frame}[fragile]{Map}
	The interface \emph{Map} is not a subinterface of \emph{Collection}.\\
	A map contains pairs of key and value. Each key refers to a value. 
	Two keys can refer to the same value. There are not two equal keys in one map.
	\emph{Map} is part of the package \texttt{java.util}.
	\vfill
	\begin{lstlisting}[basicstyle=\ttfamily\scriptsize]
	public static void main (String[] args) {
	
	Map<Integer, String> map = 
	new HashMap<Integer, String>();
	
	map.put(23, "foo");
	map.put(28, "foo");
	map.put(31, "bar");
	map.put(23, "bar"); // "bar" replaces "foo" for key = 23
	
	System.out.println(map);
	// prints: {23=bar, 28=foo, 31=bar}
	}
	\end{lstlisting}
\end{frame}

\begin{frame}[fragile]{Key, Set and Values}
	You can get the set of keys from the map.
	Because one value can exist multiple times a collection is used for the values.
	\begin{lstlisting}[basicstyle=\ttfamily\scriptsize]
	public static void main (String[] args) {
	
	// [...] map like previous slide
	
	Set<Integer> keys = map.keySet();
	Collection<String> values = map.values();
	
	System.out.println(keys);
	// prints: [23, 28, 31]
	
	System.out.println(values);
	// prints: [bar, foo, bar]
	}
	\end{lstlisting}
\end{frame}

\begin{frame}[fragile]{Iterator}
	To iterate over a map use the iterator from the set of keys.
	\begin{lstlisting}[basicstyle=\ttfamily\scriptsize]
	public static void main (String[] args) {
	
	// [...] map, keys, values like previous slide    	    
	Iterator<Integer> iter = keys.iterator();
	
	while(iter.hasNext()) {
	System.out.print(map.get(iter.next()) + " ");
	} // prints: bar foo bar
	
	System.out.println(); // print a line break
	
	for(Integer i: keys) {
	System.out.print(map.get(i) + " ");
	} // prints: bar foo bar
	}
	\end{lstlisting}
\end{frame}

\begin{frame}[fragile]{Nested Maps}
	Nested maps offer storage with key pairs.
	\begin{lstlisting}[basicstyle=\ttfamily\scriptsize]
	public static void main (String[] args) {		
	
	Map<String, Map<Integer, String>> addresses = 
	new HashMap<String, Map<Integer, String>>();
	
	addresses.put("Noethnitzer Str.", 
	new HashMap<Integer, String>());
	
	addresses.get("Noethnitzer Str.").
	put(46, "Andreas-Pfitzmann-Bau");
	addresses.get("Noethnitzer Str.").
	put(44, "Fraunhofer IWU");
	}
	\end{lstlisting}
\end{frame}

\begin{frame}[fragile]{Maps and For Each}
	You can interate through the entry set of a map (available before Java 1.8)
	\begin{lstlisting}[basicstyle=\ttfamily\scriptsize]
	Map<String, String> map = ...
	for (Map.Entry<String, String> entry : map.entrySet()) {
	System.out.println("Key: " + entry.getKey() +
	", value" + entry.getValue());
	}
	\end{lstlisting}
\end{frame}

\begin{frame}{Overview}
	\begin{center}
		\begin{tabular}{ l | l }
			List & $ \bullet $ Keeps order of objects \\
			& $ \bullet $ Easily traversible \\
			& $ \bullet $ Search not effective \\
			\hline
			Set  & $ \bullet $ No duplicates \\
			& $ \bullet $ No order - still traversible \\
			& $ \bullet $ Effective searching \\
			\hline
			Map  & $ \bullet $ Key-Value storage \\
			& $ \bullet $ Search super-effective \\
			& $ \bullet $ Traversing difficult
			
		\end{tabular}
	\end{center}
\end{frame}

\begin{frame}{Easy and some more complex exercises}
	\begin{itemize}
		\item fill a map with our 10 set elements and use the index as key. Print every item in the map.
		\item remove every item from every collection step by step and dont use clear
		\item create a vending machine company. The company stores their vending machine data in a map with place (city, ...) as key and machine as value. They also have an employee list. Each employee should appear only once (use an id). Each employee has a wage and a name. It is possible to filter employees by name or wage and to return every vending machine with city when it is empty. There can be multiple results for one city too.
	\end{itemize}
	
\end{frame}


\end{document}
