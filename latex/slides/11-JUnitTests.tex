\input{../templates/course_definitions}
% This Document contains the information about this course.

% Authors of the slides
\author{Vincent Gerber, Tilman Hinnerichs}

% Name of the Course
\institute{Java Kurs}

% Fancy Logo
\titlegraphic{\hfill\includegraphics[height=1.25cm]{../templates/fsr_logo_cropped}}


\usepackage{qtree}

\title{Java}
\subtitle{JUnit Tests}
\date{\today}

\usepackage{color}

\begin{document}

\begin{frame}
	\titlepage
\end{frame}

\begin{frame}{Overview}
	\tableofcontents
\end{frame}

\section{Introduction to JUnit}
\subsection{Why to use JUnit?}
\begin{frame}[fragile]{Why to use JUnit?}
	We would like to make sure that the following code is working properly:\\
	\begin{lstlisting}
		public class Book{
			public Book(Author author){...}
			
			private int page = 1;
			private Author author;
			
			public void turnOver(int pages){
				page+=pages;
			}
			
			public void setAuthor(Author author){
				this.author = author;
			}
			
			public Author getAuthor(){
				return author;
			}
		}
	\end{lstlisting}
\end{frame}

\begin{frame}[fragile]{Why to use JUnit?}
	We modify the code as we learned last time:\\
	\begin{lstlisting}
	public class Book{
		public Book(Author author){...}
		
		private int page = 0;
		private Author author;
		
		public void turnOver(int pages){
			if(page+pages<1){throw new InvalidArgumentException();}
			page+=pages;
		}
		
		public void setAuthor(Author author){
			if(author==null){throw new NullPointerException();}
			this.author = author;
		}
		
		public Author getAuthor(){
			return author;
		}
	}
	\end{lstlisting}
\end{frame}

\begin{frame}{Why to use JUnit?}
	\large But how can we make sure that the implementation got all the methods and functionality we would like it to have (Thinking about tasks given by a client)?\\\vspace{0.5cm}
	\pause
	That is why we would like to \textbf{test} what we have got!
\end{frame}

\begin{frame}{Why to use JUnit?}
	For testing we need the following:
	
	\vspace{0.25cm}\begin{tabular}{rl}
		Test classes/code& This will contain methods to test our code. It will have to\\& be initializable by the default constructor\vspace{0.25cm}\\
		Test methods& Mathods which are marked as test methods. They will\\& have to be public, should therefor not take parameters\\& and should return void.\vspace{0.25cm}\\\vspace{0.25cm}
		Test case& A test case is some given test values with which one would like to test the given peace of software.\\\vspace{0.25cm}
		Test framework& We will be using JUnit 4 as test framework
	\end{tabular}
\end{frame}

\subsection{What to keep in mind}
\begin{frame}{What to keep in mind}
	\begin{itemize}
		\item Use an extra test class seperated from the rest of the code
		\item You can name you test class how you want to, but *Test is common
		\item Your cases should test atomary parts of your code \textbf{AND} how those parts are working together
		\item Do not even try to test your compiler with this
	\end{itemize}
\end{frame}


\subsection{How to use and build test classes}
\begin{frame}{Assert}
	Importing org.junit.Assert.* we can use the assert method:\\
	\vspace{0.25cm}
	\begin{tabular}{rl}
		assertEqual(Object exp, Object act)&Tests whether two Objects are equal\\
		assertNotNull(Object object)&Object is not null?\\
		assertSame(Object exp, Object act)&Objects same\\
		assertNotSame(Object unexp, Object act)&Objects not identical\\
		assertTrue(boolean condition)&Tests condition\\
		assertFalse(boolean condition)&Tests condition\\
	\end{tabular}
\end{frame}

\begin{frame}{Annotations}
	We can also put several annotations in front of our tests (JUnit 4):\\
	\vspace{0.25cm}
	\begin{tabular}{rl}
		@Test&Mark the following method as test case\\
		@Test(expected=\dots Exception.class)&Mark that exception is espected\\
		@Before&Will be executed \textbf{before} every test method\\
		@After&Will be executed \textbf{after} every test method\\
		
	\end{tabular}
\end{frame}


\begin{frame}[fragile]{A simple test class}
	\begin{lstlisting}
		import java.util.*;
		import org.junit.*;
		
		public class BookTest{
			//NO CONSTRUCTOR FOR TEST CLASSES!
			@Test
			public void turnOverTest(){
				Author douglas = new Author("Douglas Adams");
				Book thgttg = new Book(douglas);
				book.turnOver(5);
				assertEquals(6,book.getPage);
			}
		}
	\end{lstlisting}
\end{frame}

\begin{frame}[fragile]{Using the magic of JUnit}
	\begin{lstlisting}
		public class BookTest{
			private Book book;
			private Author author
			@Before
			public void setUp(){
				autor = new Author("Douglas Adams");
				book = new Book(douglas);	
			}
			@Test
			public void turnOverTest(){
				book.turnOver(5);
				assertEquals(6,book.getPage);
			}
		}
	\end{lstlisting}
\end{frame}

\begin{frame}[fragile]{Testing to throw exceptions}
	\begin{lstlisting}
		public class BookTest{
			private Book book;
			private Author author
			
			@Before
			public void setUp(){
				autor = new Author("Douglas Adams");
				book = new Book(douglas);	
			}
			
			@Test (expected = InvalidArgumentException.class)
			public void turnOverTest(){
				book.turnOver(-5);
			}
		}
	\end{lstlisting}
\end{frame}

\end{document}