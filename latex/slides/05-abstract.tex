\input{../templates/course_definitions}
% This Document contains the information about this course.

% Authors of the slides
\author{Vincent Gerber, Tilman Hinnerichs}

% Name of the Course
\institute{Java Kurs}

% Fancy Logo
\titlegraphic{\hfill\includegraphics[height=1.25cm]{../templates/fsr_logo_cropped}}


\title{Java}
\subtitle{Abstract}
\date{\today}

\usepackage{tabularx}

\begin{document}

\begin{frame}
	\titlepage
\end{frame}

\begin{frame}
	\tableofcontents
\end{frame}

\section{An Addtition to Control Statements: Switch-Statements}
\begin{frame}[fragile]{Differentiate}
	A common problem:
	\begin{lstlisting}[basicstyle=\ttfamily\scriptsize]
	public static void main (String[] args) {
	
	int address = 2;
	
	if (address == 1) {
	System.out.println("Dear Sir,");	    
	} else if (address == 2) {
	System.out.println("Dear Madam,");		    
	} else if (address == 4) {
	System.out.println("Dear Friend,");		    
	} else {
	System.out.println("Dear Sir/Madam,");	
	}
	}
	\end{lstlisting}
\end{frame}
\begin{frame}[fragile]{Differentiate with Switch}
	How to write it using the Switch-statement:
	\begin{lstlisting}[basicstyle=\ttfamily\scriptsize]
	public static void main (String[] args) {
	
	int address = 2;
	
	switch(address) {
	case 1:
	System.out.println("Dear Sir,");
	break;
	case 2:
	System.out.println("Dear Madam,");
	break;
	case 4:
	System.out.println("Dear Friend,");
	break; 
	default:
	System.out.println("Dear Sir/Madam,");
	break;
	}
	}
	\end{lstlisting}
\end{frame}
\begin{frame}[fragile]{Differentiate with Switch}
	Depending on a variable you can switch the execution paths using the keyword \textbf{switch}.
	This works with \texttt{int}, \texttt{char} and \texttt{String}. \\
	\vfill
	The variable is compared 
	with the value following the keyword case.
	% Strings are not compared - dot.equal
	If they are equal the program will enter the corresponding case block.
	If nothing fits the program will enter the default block.
	\begin{lstlisting}[basicstyle=\ttfamily\scriptsize]
	public static void main (String[] args) {
	switch(intVariable) {
	case 1:
	doSomething();
	break;
	default:
	doOtherThings();
	break;
	}
	}
	\end{lstlisting}
\end{frame}

\begin{frame}[fragile]{Exercises}	
	\begin{itemize}
		\item Create a simple calculator with +,-,/ and *
		\vspace{0.5cm}
		\item Extend our wheel class with a pressure state function (low, medium, full,...)
	\end{itemize}
\end{frame}

\section{Abstract classes and methods}
\subsection{}
\begin{frame}[fragile]{Abstract Class}
	The keyword \textbf{abstract} denotes an abstract class.
	\vfill
	\begin{lstlisting}
	public abstract class AbstractExample {
	
	}	
	\end{lstlisting}
	\vfill
    \begin{itemize}
	\item You can not create objects from an abstract class.\\
	\item Abstract classes can extend other abstract classes and can implement interfaces \footnote[1]{Interfaces will be discussed later}.\\
	\item Abstract classes can be extended by normal and abstract classes.
    \end{itemize}
\end{frame}

\begin{frame}[fragile]{An Example}
	We want to model some furniture. \\
	\vspace{0.5cm}
	Hence, we want some chairs, tables, \dots\ as classes which extend the furniture class (UML diagram on whiteboard)\\
	\vspace{0.25cm}
	We do \textbf{NOT} want an object of type furniture. In this case we label the class furniture as abstract.
\end{frame}

\begin{frame}[fragile]{Methods}
	An abstract class may has concrete methods and may has abstract methods.
	\begin{lstlisting}
	public abstract class AbstractExample {
	
	    public void printHello() {
	        System.out.println("Hello");	    
	    }
	    
	    public abstract String getName();
	}	
	\end{lstlisting}
	An abstract method forces the class to be abstract as well. \\
	%\emph{Methods in an interface are also abstract but not denoted as such.}
\end{frame}

\begin{frame}[fragile]{Subclasses}
	The subclass has to implement abstract methods or has to be abstract as well.
	All concrete methods will be regularly inherited.
	\begin{lstlisting}[escapechar=!]
	public class Example extends AbstractExample {
	    
	    @Override
	    public String getName() {
	        return "Example";	    
	    }
	}	
	\end{lstlisting}
\end{frame}

%TODO more text
\begin{frame}{Why using Abstract?}
	\begin{itemize}
		\item Lets you build a template for the children classes
		\vspace{0.5cm}
		\item Lets you add functionality classes
		\vspace{0.5cm}
		\item Able to avoid redundancy of code in different classes with similar functionality
		\vspace{0.5cm}
		\item Ability to specify default implementations of methods
	\end{itemize}
\end{frame}

\section{Interfaces}
\subsection{Overview}
\begin{frame}{Interfaces}
	An \textbf{interface} is a well defined set of constants and methods a class have to \textbf{implement}.
	\vfill
	$\implies$ Makes the handling of other objects much easier
	\vfill
	\pause
	For Example: A post office offers to ship letters, postcards and packages. With an interface
	\emph{Trackable} you can collect the positions unified. 
	It is not important how a letter calculates its position.
	%TODO rethink the following sentence
	It is important that the letter communicate its position through the methods from the interface.
	%vfill
	%conclusion: Seperation from definition and implementation. 
	%The definition is ?denoted by? the interface.
\end{frame}
\subsection{Example}
\begin{frame}[fragile]{Interface Trackable}
	An interface contains method signatures. A signature is the definition of a method without the implementation.
	\begin{lstlisting}
	public interface Trackable {
	
	public int getStatus(int identifier);
	
	public Position getPosition(int identifier);
	}
	\end{lstlisting}
	Note: The name of an interface often (not always) ends with the suffix \emph{-able}.
\end{frame}
\begin{frame}[fragile]{Letter implements Trackable}
	\begin{lstlisting}
	public class Letter implements Trackable {
	
	public Position position;
	private int identifier;
	
	public int getStatus(int identifier) {
	return this.identifier;
	}
	
	public Position getPosition(int identifier) {
	return this.position;
	}
	}
	\end{lstlisting}
	The classes \emph{Postcard} and \emph{Package} also implement the interface \emph{Trackable}.
\end{frame}

\begin{frame}[fragile]{Overview}
	\huge Time for some UML!
\end{frame}

\begin{frame}[fragile]{Access through an Interface}
	\begin{lstlisting}
	public static void main(String[] args) {
	
	Trackable letter_1 = new Letter();
	Trackable letter_2 = new Letter();
	Trackable postcard_1 = new Postcard();
	Trackable package_1 = new Package();
	
	letter_1.getPosition(2345);
	postcard_1.getStatus(1234);
	}
	\end{lstlisting}
\end{frame}

\subsection{Multiple Interfaces}
\begin{frame}[fragile]{Two Interfaces}
	A class can implement multiple interfaces.
	\vfill
	\begin{lstlisting}[basicstyle=\ttfamily\scriptsize]
	public interface Buyable {
	
	// constant
	public float tax = 1.19f;
	
	public float getPrice();
	}
	\end{lstlisting}
	\begin{lstlisting}[basicstyle=\ttfamily\scriptsize]
	public interface Trackable {
	
	public int getStatus(int identifier);
	
	public Position getPosition(int identifier);
	}
	\end{lstlisting}
\end{frame}

\begin{frame}[fragile]{Postcard implements Buyable and Trackable}
	\begin{lstlisting}[basicstyle=\ttfamily\scriptsize]
	public class Postcard implements Buyable, Trackable {
	
	public Position position;
	private int identifier;
	private float priceWithoutVAT;
	
	public float getPrice() {
	return priceWithoutVAT * tax;
	}
	
	public int getStatus(int identifier) {
	return this.identifier;
	}
	
	public Position getPosition(int identifier) {
	return this.position;
	}
	}
	\end{lstlisting}
\end{frame}

\begin{frame}[fragile]{Access multiple Interfaces}
	\begin{lstlisting}
	public static void main(String[] args) {
	
	Trackable postcard_T = new Postcard();
	Postcard postcard_P = new Postcard();
	Buyable postcard_B = new Postcard();
	
	postcard_T.getStatus(1234);
	postcard_B.getPrice();
	postcard_P.getStatus(1234);
	postcard_P.getPrice();
	}
	\end{lstlisting}
	\texttt{postcard\_P} can access both interfaces.\\
	\texttt{postcard\_T} can access Trackable.\\
	\texttt{postcard\_B} can access Buyable.
\end{frame}
%deault methods and static methods	


\begin{frame}{Abstract Class vs. Interface}
	\begin{tabularx}{\textheight}{l|l}
		\textbf{Abstract class} & \textbf{Interface} \\
		$\bullet$ Template for children classes & $\bullet$ Template for implementations \\
		$\bullet$ Able to add functionality & $\bullet$ \textbf{NO} functionality in interface itself \\
		$\bullet$ No multi-inheritance & $\bullet$ Sort of multi-inheritance \\
		$\bullet$ Default implementations of methods & $\bullet$ \textbf{NO} default implementation
	\end{tabularx}\\
	\vspace{0.5cm}
	The usage of both can overlap.
\end{frame}

\begin{frame}[fragile]{Exercises}
	Let's start with a small drawing application
	\begin{itemize}
		\item Create a base class for every drawable object (with functions like draw, getPosition, ...)
		\vspace{0.5cm}
		\item Create two different drawable classes (e.g circle, rectangle)
		\vspace{0.5cm}
		\item Implement an interface Size with getVolume and getArea as functions. Every Drawable should implement this interface
		\vspace{0.5cm}
		\item Declare a function with an object as input and the volume and area as output.
	\end{itemize}
\end{frame}

%TODO Common Errors
%\begin{frame}{Static vs. Abstract Methods}
%	Do not mix up static and abstract methods.
%\end{frame}
\end{document}